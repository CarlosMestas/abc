\documentclass[12pt]{article}
\usepackage[margin=0.75in]{geometry}
\geometry{a4paper}
\usepackage[T1]{fontenc} % Support Icelandic Characters
\usepackage[utf8]{inputenc} % Support Icelandic Characters
\usepackage{graphicx} % Support for including images
\usepackage{hyperref} % Support for hyperlinks
\usepackage{wrapfig}

\usepackage{algorithm}
\usepackage{algorithmicx}
\usepackage{listings}
\usepackage{color}
\usepackage{siunitx}

\usepackage[svgnames]{xcolor}

\usepackage[spanish]{babel}
\usepackage[latin1]{inputenc}
\usepackage[usenames]{color}

\definecolor{mGreen}{rgb}{0,0.6,0}
\definecolor{mGray}{rgb}{0.5,0.5,0.5}
\definecolor{mPurple}{rgb}{0.58,0,0.82}
\definecolor{backgroundColour}{rgb}{0.95,0.95,0.92}

\usepackage{listings}             % Include the listings-package
\lstdefinestyle{CStyle}{
    backgroundcolor=\color{backgroundColour},   
    commentstyle=\color{mGreen},
    keywordstyle=\color{blue},
    numberstyle=\tiny\color{mGray},
    stringstyle=\color{red},
    basicstyle=\footnotesize,
    breakatwhitespace=false,         
    breaklines=true,                 
    captionpos=b,                    
    keepspaces=true,                 
    numbers=left,                    
    numbersep=5pt,                  
    showspaces=false,                
    showstringspaces=false,
    showtabs=false,                  
    tabsize=2,
    language=C
}


%------------------------------------------------------------------
% TITLE
%------------------------------------------------------------------

\title{
\centerline{
    \includegraphics[width=75mm]{unsa.png}}
    \vspace{0.5 cm}
        Programación de Sistemas - Laboratorio - Grupo B
        \\
        \\
        \\
        \textbf{Práctica Laboratorio N6} 
        \large  
        \\
        %SC-T-718-ATSR,Automatic Speech Recognition, 2019-1 
        %\\ 
        \small Universidad Nacional de San Agustín - Escuela Profesional de Ingeniería de Sistemas, Arequipa, Perú 
  }

\author{
    Carlos Alberto Mestas Escarcena
    \\
    \texttt{cmestas@unsa.edu.pe}
}

\date{Junio 2020}

\begin{document}

\maketitle

El desarrollo de este informe se puede encontrar en el repositorio de \textcolor{blue}{
    \href{https://github.com/CarlosMestas/TC_A_6_Carlos_Mestas}{GitHub}}.
    
%%%%1
\section{Ejercicio 1}

\begin{lstlisting}[language=bash,frame=single,style=CStyle]
/*
  Ejercicio 1: Explicacion y documentacion del codigo fuente
  AUTOR: Luis García
  CARRERA: Ingeniería en Computación
*/

#include <stdlib.h>
#include <iostream>
#include <fstream>
#include <windows.h>
#include <string.h>
#include <stdio.h>
#include <stdlib.h>

using namespace std;

// DEFINIENDO LAS VARIABLES NECESARIAS
string nombre, auxnombre, semestre, edad;
int clave = 0, auxclave = 0;
char grupo[10];
char opca;
bool encontrado = false;

/*
  Funcion que nos permite agregar alumnos
*/
void altas() {
    // Variables de la biblioteca fstream para el manejo de archivos
    ofstream escritura;
    ifstream consulta;
    do {
        escritura.open("alumnos.txt", ios::out | ios::app);// crea y escribe, si ya tiene texto une al final del archivo
        consulta.open("alumnos.txt", ios::in);// solamente consulta o lee usando la variable sobre el archivo físico alumnos.txt
        if (escritura.is_open() && consulta.is_open()) {
            // Variable que va a servir para indicar si el codigo de un alumno ya existe o no
            bool repetido = false;
            cout << "\n";
            cout << "\tIngresa la clave del alumno:    ";
            cin >> auxclave;
            // A continuación se aplica el tipo de lectura de archivos secuencial
            consulta >> clave;
            // Realizara todo el bucle en caso de que estemos ingresando un codigo ya existente en otro alumno
            while (!consulta.eof()) {
                consulta >> nombre >> semestre >> grupo >> edad;
                if (auxclave == clave) {
                    cout << "\t\tYa existe la clave del alumno...\n";
                    repetido = true;
                    break;
                }
                consulta >> clave;
            }
            // Cuando el codigo es nuevo si se continua a ingresar los datos del nuevo alumno
            if (repetido == false) {
                cout << "\tIngresa el nombre del alumno:   ";
                cin >> nombre;
                cout << "\tIngresa el semestre del alumno: ";
                cin >> semestre;
                cout << "\tIngresa el grupo del alumno:    ";
                cin >> grupo;
                cout << "\tIngresa la edad del alumno:     ";
                cin >> edad;
                // ESCRIBIENDO LOS DATOS CAPTURADOS POR EL USUARIO EN EL ARCHIVO
                escritura << auxclave << " " << nombre << " " << semestre << " " << grupo << " " << edad << endl;
                cout << "\n\tRegistro agregado...\n";
            }
            // Se consulta si se desea agregar mas altas
            cout << "\n\tDeseas ingresar otro alumno? (S/N): ";
            cin >> opca;

        }
        // En el caso de que no se pueda abrir el archivo ni crearlo
        else {
            cout << "El archivo no se pudo abrir \n";
        }
        escritura.close();
        consulta.close();

    } while (opca == 'S' or opca == 's');

}
/*
  Funcion que nos permite eliminar alumnos
*/
void bajas(){
    ofstream aux;
    ifstream lectura;
    // Variable que nos indicara si se encontro la clave de un alumno o no
    encontrado = false;
    // Se va a crear un archivo auxiliar donde se van a copiar los alumnos mientras se vayan eliminando
    // Este archivo reemplazara al alumnos.txt existente
    aux.open("auxiliar.txt", ios::out);
    lectura.open("alumnos.txt", ios::in);

    if (aux.is_open() && lectura.is_open()) {
        cout << "\n";
        cout << "\tIngresa la clave del alumno que deseas eliminar: ";
        cin >> auxclave;
        // De nuevo se aplica el tipo de lectura de archivos secuencial, esto quiere decir que lee campo por campo hasta
        // hasta llegar al final del archivo gracias a la función .eof()
        lectura >> clave;
        while (!lectura.eof()) {
            lectura >> nombre >> semestre >> grupo >> edad;
            if (auxclave == clave) {
                encontrado = true;
                cout << "\n";
                cout << "\tClave:    " << clave << endl;
                cout << "\tNombre:   " << nombre << endl;
                cout << "\tSemestre: " << semestre << endl;
                cout << "\tGrupo:    " << grupo << endl;
                cout << "\tEdad:     " << edad << endl;
                cout << "\t________________________________\n\n";
                cout << "\tRealmente deseas eliminar el registro actual (S/N)? ---> ";
                cin >> opca;
                if (opca == 'S' or opca == 's') {
                    cout << "\n\n\t\t\tRegistro eliminado...\n\n\t\t\a";
                }
                else {
                    aux << clave << " " << nombre << " " << semestre << " " << grupo << " " << edad << endl;
                }
            }
            else {
                aux << clave << " " << nombre << " " << semestre << " " << grupo << " " << edad << endl;
            }
            lectura >> clave;
        }
    }
    else {
        cout << "\n\tEl archivo no se pudo abrir \n";
    }

    if (encontrado == false) {
        cout << "\n\tNo se encontro ningun registro con la clave: " << auxclave << "\n\n\t\t\t";
    }
    aux.close();
    lectura.close();
    // Realiza el cambio de nombre del auxiliar.txt donde se encuentran los alumnos que no han sido eliminados
    remove("alumnos.txt");
    rename("auxiliar.txt", "alumnos.txt");
}

/*
  Funcion que realizara la impresion de todos los alumnos que existen actualmente
*/
void consultas(){
    // Lectura de archivos
    ifstream lectura;
    lectura.open("alumnos.txt", ios::out | ios::in);// CREA, ESCRIBE O LEE
    if (lectura.is_open()) {
        // LEYENDO Y MOSTRANDO CADA CAMPO DEL ARCHIVO DE FORMA SECUENCIAL
        lectura >> clave;
        while (!lectura.eof()) {
            lectura >> nombre >> semestre >> grupo >> edad;
            cout << "\n";
            cout << "\tClave:    " << clave << endl;
            cout << "\tNombre:   " << nombre << endl;
            cout << "\tSemestre: " << semestre << endl;
            cout << "\tGrupo:    " << grupo << endl;
            cout << "\tEdad:     " << edad << endl;
            lectura >> clave;
            cout << "\t________________________________\n";
        }

    }
    else {
        cout << "El archivo no se pudo abrir \n";
    }
    lectura.close();
}

/*
  Funcion que va a realizar una busqueda mediante el codigo de un alumno si existe lo muestra
*/
void buscar(){
    ifstream lectura;
    lectura.open("alumnos.txt", ios::out | ios::in);
    // Variable que indicara si el alumno se encuentra en los datos o no
    encontrado = false;

    if (lectura.is_open()) {
        cout << "\n\tIngresa la clave del alumno que deseas buscar: ";
        cin >> auxclave;

        lectura >> clave;
        while (!lectura.eof()) {
            lectura >> nombre >> semestre >> grupo >> edad;
            //comparar cada registro para ver si es encontrado

            if (auxclave == clave) {
                cout << "\n";
                cout << "\tClave:    " << clave << endl;
                cout << "\tNombre:   " << nombre << endl;
                cout << "\tSemestre: " << semestre << endl;
                cout << "\tGrupo:    " << grupo << endl;
                cout << "\tEdad:     " << edad << endl;
                cout << "\t________________________________\n";
                encontrado = true;
                break;
            }//fin del if mostrar encontrado

            //lectura adelantada
            lectura >> clave;
        }//fin del while
        if (encontrado == false) {
            cout << "\n\n\tNo hay registros con la clave: " << auxclave << "\n\n\t\t\t";
        }
    }
    else {
        cout << "\n\tAun no se pudo abrir el archivo...";
    }

    lectura.close();
}

/*
  Funcion que va a permitir modificar el nombre de un alumno
*/
void modificar(){
    ofstream aux;
    ifstream lectura;
    // Variable que indicara si el alumno se encuentra en los datos o no
    encontrado = false;
    // De la misma manera que se trabajo el anterior caso de eliminacion, tambien se va a necesitar
    // un archivo auxiliar donde se almacenaran los datos cambiados
    aux.open("auxiliar.txt", ios::out);
    lectura.open("alumnos.txt", ios::in);

    if (aux.is_open() && lectura.is_open()) {

        cout << "\n";
        cout << "\tIngresa la clave del alumno que deseas modificar: ";
        cin >> auxclave;

        lectura >> clave;
        while (!lectura.eof()) {
            lectura >> nombre >> semestre >> grupo >> edad;
            if (auxclave == clave) {
                encontrado = true;
                cout << "\n";
                cout << "\tClave:    " << clave << endl;
                cout << "\tNombre:   " << nombre << endl;
                cout << "\tSemestre: " << semestre << endl;
                cout << "\tGrupo:    " << grupo << endl;
                cout << "\tEdad:     " << edad << endl;
                cout << "\t________________________________\n\n";
                cout << "\tIngresa el nuevo nombre del alumno con la clave: " << auxclave << "\n\n\t---> ";
                cin >> auxnombre;

                aux << clave << " " << auxnombre << " " << semestre << " " << grupo << " " << edad << endl;
                cout << "\n\n\t\t\tRegistro modificado...\n\t\t\a";
            }
            else {
                aux << clave << " " << nombre << " " << semestre << " " << grupo << " " << edad << endl;
            }
            lectura >> clave;
        }
    }
    else {
        cout << "\n\tEl archivo no se pudo abrir \n";
    }

    if (encontrado == false) {
        cout << "\n\tNo se encontro ningun registro con la clave: " << auxclave << "\n\n\t\t\t";
    }
    aux.close();
    lectura.close();
    // Se realiza el cambio 
    remove("alumnos.txt");
    rename("auxiliar.txt", "alumnos.txt");
}

/*
  Funcion principal
*/
int main(){
    // Cambio de color de la consola
    system("color f0");
    int opc;
    do {
        // Limpia la consola
        system("cls");
        // Menu de opciones
        cout << "\n\tManejo de archivos en c++\n\n";
        cout << "\n\t1.-Altas";
        cout << "\n\t2.-Bajas";
        cout << "\n\t3.-Consultas";
        cout << "\n\t4.-Buscar un registro";
        cout << "\n\t5.-Modificaciones";
        cout << "\n\t6.-Salir";
        cout << "\n\n\tElige una opcion:  ";
        cin >> opc;
        switch (opc){
        case 1: {
            system("cls");
            altas();
            cout << "\n\t\t";
            system("pause");
            break;
        }
        case 2: {
            system("cls");
            bajas();
            cout << "\n\t\t";
            system("pause");
            break;
        }

        case 3: {
            system("cls");
            consultas();
            cout << "\n\t\t";
            system("pause");
            break;
        }
        case 4: {
            system("cls");
            buscar();
            cout << "\n\t\t";
            system("pause");
            break;
        }
        case 5: {
            system("cls");
            modificar();
            cout << "\n\t\t";
            system("pause");

            break;
        }
        case 6: {
            cout << "\n\n\tHa elegido salir...\n\n\t\t"; system("pause");
            break;
        }
        default: {
            cout << "\n\n\tHa elegido una opcion inexistente...\n\n\t\t"; system("pause");
            break;
        }
        }//fin switch

    } while (opc != 6);
    // Limpia la consola
    system("cls");
    return 0;
}
\end{lstlisting}

\section{Ejercicio 2}

\begin{lstlisting}[language=bash,frame=single,style=CStyle]
// Exercise_2.cpp : Desarrollo del ejercicio 2
// Curso: Programacion de Sistemas
// Laboratorio: B
// Autor: Carlos Alberto Mestas Escarcena
// Fecha: 19/06/2020

#include <stdlib.h>
#include <iostream>
#include <fstream>
#include <windows.h>
#include <string.h>
#include <stdio.h>
#include <stdlib.h>
#include <string>

using namespace std;

// Definiendo las variables

struct Tdato
{
    int b;
    char s[100];
};
int x, n, a[10] = {1,2,3,4,5,6,7,8,9,0};
double f;
char nombre[] = "Ejercicios ficheros binarios";
Tdato p;
ifstream f1;
ofstream f2;

/*
  Funcion principal
*/
int main(){
    // Cambio de color de la consola
    system("color f0");
    f1.open("entrada.dat", ios::binary);
    f2.open("salida.dat", ios::binary);
    x = 7;
    f = 5.125;
    p.b = 123;
    strcpy_s(p.s, "Carlos");
    if (f2.is_open()) {
        f2 << x << endl;
        f2 << f << endl;
        for (int i = 0; i < 5; i++) {
            f2 << a[i];
        }
        f2 << endl;
        for (int i = 0; i < 10; i++) {
            f2 << nombre[i];
        }
        f2 << endl;
        f2 << p.b << " " << p.s << endl;

    }
    else {
        cout << "Error al escribir en el fichero";
    }
    string c;
    string c1, c2, c3, c4, c5, c6;
    if (f1.is_open()) {
        int i = 1;
        while (!f1.eof()) {
            f1 >> c;
            if (i == 1) {
                c1 = c;
                i++;
            }
            else if (i == 2) {
                c2 = c;
                i++;
            }
            else if (i == 3) {
                c3 = c;
                i++;
            }
            else if (i == 4) {
                c4 = c;
                i++;
            }
            else if (i == 5) {
                c5 = c;
                i++;
            }
            else {
                c6 = c;
                i++;
            }
        }
    }
    else {
        cout << "Error al leer el fichero";
    }
    // Asignamos los distintos valores del fichero de lectura a los variables
    x = std::stoi(c1);
    f = std::stof(c2);
    for (int j = 0; j < 5; j++) {
        a[j] = (int)(c3.at(j)) - 48;
    }
    for (int j = 0; j < 10; j++) {
        nombre[j] = (c4.at(j));
    }
    p.b = std::stoi(c5);
    strcpy_s(p.s, c6.c_str());
    // Impresiones de las variables
    cout << "Impresion de los valores de lectura: " << endl;
    cout << "x = " << x << endl;
    cout << "f = " << f << endl;
    cout << "5 primeros valores del arreglo: ";
    for (int j = 0; j < 5; j++) {
        cout << a[j];
    }
    cout << endl;
    for (int j = 0; j < 10; j++) {
        cout << nombre[j];
    }
    cout << endl;
    cout << p.b << "\t" << p.s;

    return 0;
}
\end{lstlisting}

\newpage
\clearpage

Valores de entrada que se van a leer y almacenar en las variables:

\begin{figure}[h]
    \centering
    \includegraphics[width=1\textwidth]{images/Capture002B.PNG}
    \caption{Ejercicio 2 - entrada.dat}
\end{figure}

Valore que se almacenan mediante código:

\begin{figure}[h]
    \centering
    \includegraphics[width=1\textwidth]{images/Capture002A.PNG}
    \caption{Ejercicio 2 - salida.dat}
\end{figure}

Ejemplo de impresión de las variables almacenadas:

\begin{figure}[h]
    \centering
    \includegraphics[width=1\textwidth]{images/Capture002C.PNG}
    \caption{Ejercicio 2 - Ejecución}
\end{figure}


\newpage
\clearpage

\section{Ejercicio 3}
\begin{lstlisting}[language=bash,frame=single,style=CStyle]
// Exercise_3.cpp : Desarrollo del ejercicio 3
// Curso: Programacion de Sistemas
// Laboratorio: B
// Autor: Carlos Alberto Mestas Escarcena
// Fecha: 19/06/2020

#include <iostream>
#include <string>
#include <fstream>
#include <stdio.h>
#include <string.h>

using namespace std;

void write() {
    ofstream write;
    write.open("phrases.txt", ios::out | ios::app);// crea y escribe, si ya tiene texto une al final del archivo
    if (write.is_open()) {
        char tmp[60];
        cout << "\nIntroduzca una frase: \n";
        gets_s(tmp);
        // Si ingresamos vacio
        while (strcmp(tmp, "") != 0) {
            write << tmp << "\n";
            cout << "\nIntroduzca una frase: \n";
            gets_s(tmp);
        }
    }
    // En el caso de que no se pueda abrir el archivo ni crearlo
    else {
        cout << "El archivo no se pudo abrir \n";
    }
    write.close();
}

void read() {
    ifstream read;
    read.open("phrases.txt", ios::out | ios::in);
    int key = 0;
    string phraseTmp;
    if (read.is_open()) {
        // LEYENDO Y MOSTRANDO CADA CAMPO DEL ARCHIVO DE FORMA SECUENCIAL
        while (read) {
            char str[255];
            read.getline(str, 255);  // delim defaults to '\n'
            if (read){
                cout << str << endl;
                cout << "--------------------------------------------------" << endl;
            }
        }

    }
    else {
        cout << "El archivo no se pudo abrir \n";
    }
    read.close();
}

int main() {
    system("color f0");
    write();
    read();
    exit(0);
    return 0;
}
\end{lstlisting}

\begin{figure}[h]
    \centering
    \includegraphics[width=1\textwidth]{images/Capture003A.PNG}
    \caption{Ejercicio 3 - Ejecución}
\end{figure}

\section{Ejercicio 4}

\begin{lstlisting}[language=bash,frame=single,style=CStyle]
// Exercise_4.cpp : Desarrollo del ejercicio 4
// Curso: Programacion de Sistemas
// Laboratorio: B
// Autor: Carlos Alberto Mestas Escarcena
// Fecha: 19/06/2020

#include <iostream>
#include <string>
#include <fstream>
#include <stdio.h>
#include <string.h>

using namespace std;

void readFile() {
    ifstream read;
    string nameFile;
    char tmp0[25];
    cout << "Ingrese nombre del archivo:";
    cin >> nameFile;
    gets_s(tmp0);
    cout << endl;
    read.open(nameFile, ios::in);
    string phraseTmp;
    if (read.is_open()) {
        int i = 0;
        while (read && i < 25) {
            char str[255];
            read.getline(str, 255);
            i++;
            cout << i << "\t";
            if (read) 
                cout << str << endl;
            
            if (i == 25) {
                cout << endl;
                char tmp[1];
                cout << "Presione ENTER si desea continuar con la lectura: \n";
                gets_s(tmp);
                if (strcmp(tmp, "") == 0)
                    i = 0;
                else {
                    cout << "\n Programa finalizado";
                    exit(0);
                }
                    
            }

        }
        cout << "---------------Final del archivo-------------------" << endl;
    }
    else {
        cout << "El archivo no se pudo abrir o no existe\n";
    }
    read.close();
}

int main() {
    system("color f0");
    readFile();
    exit(0);
    return 0;
}
\end{lstlisting}

\newpage
\clearpage

\begin{figure}[h]
    \centering
    \includegraphics[width=1\textwidth]{images/Capture004A.PNG}
    \caption{Ejercicio 4 - Ejecución parte I}
\end{figure}

\newpage
\clearpage

\begin{figure}[h]
    \centering
    \includegraphics[width=1\textwidth]{images/Capture004B.PNG}
    \caption{Ejercicio 4 - Ejecución parte II}
\end{figure}

\newpage
\clearpage

\begin{figure}[h]
    \centering
    \includegraphics[width=1\textwidth]{images/Capture004C.PNG}
    \caption{Ejercicio 4 - Ejecución parte III}
\end{figure}

\newpage
\clearpage

\section{Ejercicio 5}

\begin{lstlisting}[language=bash,frame=single,style=CStyle]
// Exercise_5.cpp : Desarrollo del ejercicio 5
// Curso: Programacion de Sistemas
// Laboratorio: B
// Autor: Carlos Alberto Mestas Escarcena
// Fecha: 19/06/2020
// Modificacion del ejercicio 1 de autor Luis Garcia

#include <stdlib.h>
#include <iostream>
#include <fstream>
#include <windows.h>
#include <string.h>
#include <stdio.h>
#include <stdlib.h>

using namespace std;

// Definiendo las variables
string name, address, phone, email, day, month, year;
string nameT = "", auxName = "";
char grupo[100];
char opca;
bool encontrado = false;

/*
  Funcion que nos permite agregar contactos
*/
void addContact() {
    // Variables de la biblioteca fstream para el manejo de archivos
    ofstream escritura;
    ifstream consulta;
    do {
        escritura.open("agenda.dat", ios::out | ios::app);
        consulta.open("agenda.dat", ios::in);
        if (escritura.is_open() && consulta.is_open()) {
            // Variable que va a servir para indicar si el nombre de un alumno ya existe o no
            bool repetido = false;
            cout << "\n";
            cout << "\tIngresa el nombre del contacto: ";
            cin >> auxName;
            // A continuación se aplica el tipo de lectura de archivos secuencial
            consulta >> nameT;
            // Realizara todo el bucle en caso de que estemos ingresando un nombre ya existente en otro contacto
            while (!consulta.eof()) {
                consulta >> address >> phone >> email >> day >> month >> year;
                if (auxName == nameT) {
                    cout << "\t\tYa existe ese contacto en la agenda ...\n";
                    repetido = true;
                    break;
                }
                consulta >> nameT;
            }
            // Cuando el nombre es nuevo si se continua a ingresar los datos del nuevo conctacto
            if (repetido == false) {
                cout << "\tIngresa la direccion del contacto: ";
                cin >> address;
                cout << "\tIngresa el numero de celular del contacto: ";
                cin >> phone;
                cout << "\tIngresa el correo del contacto:     ";
                cin >> email;
                cout << "\tIngresa el dia de nacimiento del contacto: ";
                cin >> day;
                cout << "\tIngresa el mes de nacimiento del contacto: ";
                cin >> month;
                cout << "\tIngresa el anio de nacimiento del contacto: ";
                cin >> year;
                // Escribe los datos del contacto en el archivo
                escritura << auxName << " " << address << " " << phone << " " << email << " " << day << " " << month << " " << year << endl;
                cout << "\n\tContacto agregado correctamente...\n";
            }
            // Se consulta si se desea agregar mas contactos
            cout << "\n\tDeseas ingresar otro contacto? (S/N): ";
            cin >> opca;
        }
        // En el caso de que no se pueda abrir el archivo ni crearlo
        else {
            cout << "El archivo no se pudo abrir \n";
        }
        escritura.close();
        consulta.close();

    } while (opca == 'S' or opca == 's');

}

/*
  Funcion que realizara la impresion de todos los contactos existentes
*/
void showContacts() {
    // Lectura de archivos
    ifstream lectura;
    lectura.open("agenda.dat", ios::out | ios::in);
    if (lectura.is_open()) {
        lectura >> name;
        while (!lectura.eof()) {
            lectura >> address >> phone >> email >> day >> month >> year;
            cout << "\n";
            cout << "\tNombre:    " << name << endl;
            cout << "\tDireccion:    " << address << endl;
            cout << "\tTelefono:    " << phone << endl;
            cout << "\tCorreo:    " << email << endl;
            cout << "\tFecha de nacimiento:    " << day << " / "<< month<< " / "<< year<<endl;
            lectura >> nameT;
            cout << "\t________________________________\n";
        }

    }
    else {
        cout << "El archivo no se pudo abrir \n";
    }
    lectura.close();
}

/*
  Funcion que va a realizar una busqueda mediante el nombre de un contacto si existe lo muestra
*/

void searchConctact() {
    ifstream lectura;
    lectura.open("agenda.dat", ios::out | ios::in);
    // Variable que indicara si el contacto se encuentra en la agenda o no
    encontrado = false;
    if (lectura.is_open()) {
        cout << "\n\tIngresa el nombre del contacto que desea buscar: ";
        cin >> auxName;
        lectura >> nameT;
        while (!lectura.eof()) {
            lectura >> address >> phone >> email >> day >> month >> year;
            // Comparar cada registro para ver si es encontrado
            if (auxName == nameT) {
                cout << "\n";
                cout << "\tNombre:    " << nameT << endl;
                cout << "\tDireccion:    " << address << endl;
                cout << "\tTelefono:    " << phone << endl;
                cout << "\tCorreo:    " << email << endl;
                cout << "\tFecha de nacimiento:    " << day << " / " << month << " / " << year << endl;
                lectura >> nameT;
                cout << "\t________________________________\n";
                encontrado = true;
                break;
            }
            lectura >> nameT;
        }
        if (encontrado == false) {
            cout << "\n\n\tNo hay ningun contacto con el nombre: " << auxName << "\n\n\t\t\t";
        }
    }
    else {
        cout << "\n\tAun no se pudo abrir el archivo...";
    }

    lectura.close();
}

/*
  Funcion principal
*/
int main(){
    // Cambio de color de la consola
    system("color f0");
    int opc;
    do {
        // Limpia la consola
        system("cls");
        // Menu de opciones
        cout << "\n\tMI AGENDA\n";
        cout << "\n\t1. Agregar contacto";
        cout << "\n\t2. Mostrar todos los contactos";
        cout << "\n\t3. Buscar un contacto";
        cout << "\n\t4. Salir";
        cout << "\n\n\tElige una opcion:  ";
        cin >> opc;
        switch (opc){
        case 1: {
            system("cls");
            addContact();
            cout << "\n\t\t";
            system("pause");
            break;
        }
        case 2: {
            system("cls");
            showContacts();
            cout << "\n\t\t";
            system("pause");
            break;
        }
              
        case 3: 
            system("cls");
            searchConctact();
            cout << "\n\t\t";
            system("pause");
            break;       
        case 4: {
            cout << "\n\n\tHa elegido salir...\n\n\t\t"; 
            system("pause");
            break;
        }
        default: {
            cout << "\n\n\tHa elegido una opcion que no es encuentra en el menu ...\n\n\t\t"; 
            system("pause");
            break;
        }
        }

    } while (opc != 4);
    system("cls");
    return 0;
}
\end{lstlisting}

\begin{figure}[h]
    \centering
    \includegraphics[width=0.5\textwidth]{images/Capture005A.PNG}
    \caption{Ejercicio 5 - Ejecución}
\end{figure}

\begin{figure}[h]
    \centering
    \includegraphics[width=0.7\textwidth]{images/Capture005B.PNG}
    \caption{Ejercicio 5 - Ejecución}
\end{figure}

\begin{figure}[h]
    \centering
    \includegraphics[width=0.7\textwidth]{images/Capture005C.PNG}
    \caption{Ejercicio 5 - Ejecución}
\end{figure}

\begin{figure}[h]
    \centering
    \includegraphics[width=0.7\textwidth]{images/Capture005D.PNG}
    \caption{Ejercicio 5 - Ejecución}
\end{figure}

\begin{figure}[h]
    \centering
    \includegraphics[width=0.7\textwidth]{images/Capture005E.PNG}
    \caption{Ejercicio 5 - Ejecución}
\end{figure}
\end{document}


